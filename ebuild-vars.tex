\chapter{Ebuild-defined Variables}
\label{sec:ebuild-vars}

\note This section describes variables that may or must be defined by ebuilds. For
variables that are passed from the package manager to the ebuild, see section~\ref{sec:ebuild-env-vars}.

If any of these variables are set to invalid values, or if any of the mandatory variables are
undefined, the package manager's behaviour is undefined; ideally, an error in one ebuild should not
prevent operations upon other ebuilds or packages.

\section{Metadata invariance}
\label{sec:metadata-invariance}

All ebuild-defined variables discussed in this chapter must be defined independently of
any system, profile or tree dependent data, and must not vary depending upon the ebuild
phase. In particular, ebuild metadata can and will be generated on a different system from that upon
which the ebuild will be used, and the ebuild must generate identical metadata every time it
is used.

Globally defined ebuild variables without a special meaning must similarly not rely upon
variable data.

\section{Mandatory Ebuild-defined Variables}

All ebuilds must define at least the following variables:

\begin{description}
\item[DESCRIPTION] A short human-readable description of the package's purpose. May be defined by an
    eclass. Must not be empty.
\item[SLOT] The package's slot. Must be a valid slot name, as per section~\ref{sec:slot-names}.
    May be defined by an eclass. Must not be empty.

    In EAPIs shown in table~\ref{tab:slot-deps-table} as supporting sub-slots, the \t{SLOT} variable
    may contain an optional sub-slot part that follows the regular slot and is delimited by a \t{/}
    character. The sub-slot must be a valid slot name, as per section~\ref{sec:slot-names}.
    The sub-slot is used to represent cases in which an upgrade to a new version of a package with
    a different sub-slot may require dependent packages to be rebuilt. When the sub-slot part is
    omitted from the \t{SLOT} definition, the package is considered to have an implicit sub-slot
    which is equal to the regular slot.
\end{description}

\section{Optional Ebuild-defined Variables}

Ebuilds may define any of the following variables:

\begin{description}
\item[EAPI] The EAPI\@. See below.
\item[HOMEPAGE] The URI or URIs for a package's homepage, including protocols.
    See section~\ref{sec:dependencies} for full syntax.
\item[SRC\_URI] A list of source URIs for the package. Valid protocols are \t{http://},
    \t{https://}, \t{ftp://} and \t{mirror://} (see section~\ref{sec:thirdpartymirrors} for mirror
    behaviour). Fetch restricted packages may include URL parts consisting of just a filename.
    See section~\ref{sec:dependencies} for full syntax.
\item[LICENSE] The package's license. Each text token must be a valid license name, as per
    section~\ref{sec:license-names}, and must correspond to a tree ``licenses/'' entry
    (see section~\ref{sec:licenses-dir}). See section~\ref{sec:dependencies} for full syntax.
    \label{ebuild-var-LICENSE}
\item[KEYWORDS] A whitespace separated list of keywords for the ebuild. Each token must be a valid
    keyword name, as per section~\ref{sec:keyword-names}. See section~\ref{sec:keywords} for full
    syntax.
\item[IUSE] The \t{USE} flags used by the ebuild. Any eclass that works with \t{USE} flags must
    also set \t{IUSE}, listing only the variables used by that eclass. The package manager is
    responsible for merging these values. See section~\ref{sec:use-iuse-handling} for discussion on
    which values must be listed in this variable.

    \featurelabel{iuse-defaults} In EAPIs shown in table~\ref{tab:iuse-defaults-table} as supporting
    \t{IUSE} defaults, any use flag name in \t{IUSE} may be prefixed by at most one of a plus or a
    minus sign. If such a prefix is present, the package manager may use it as a suggestion as to
    the default value of the use flag if no other configuration overrides it.
\item[REQUIRED\_USE] \featurelabel{required-use} Zero or more assertions that must be met by the
    configuration of \t{USE} flags to be valid for this ebuild. See section~\ref{sec:required-use}
    for description and section~\ref{sec:dependencies} for full syntax. Only in EAPIs listed in
    table~\ref{tab:optional-vars-table} as supporting \t{REQUIRED\_USE}.
\item[PROPERTIES] \featurelabel{properties} Zero or more properties for this package.
    See section~\ref{sec:properties} for value meanings and section~\ref{sec:dependencies} for full
    syntax. For EAPIs listed in table~\ref{tab:optional-vars-table} as having optional support,
    ebuilds must not rely upon the package manager recognising or understanding this variable in
    any way.
\item[RESTRICT] Zero or more behaviour restrictions for this package. See section~\ref{sec:restrict}
    for value meanings and section~\ref{sec:dependencies} for full syntax.
\item[DEPEND] See section~\ref{sec:dependencies}.
\item[RDEPEND] See section~\ref{sec:dependencies}. For some EAPIs, \t{RDEPEND} has special behaviour
    for its value if unset and when used with an eclass. See section~\ref{sec:rdepend-depend} for
    details.
\item[PDEPEND] See section~\ref{sec:dependencies}.
\end{description}

\ChangeWhenAddingAnEAPI{6}
\begin{centertable}{EAPIs supporting \t{IUSE} defaults}
    \label{tab:iuse-defaults-table}
    \begin{tabular}{ll}
      \toprule
      \multicolumn{1}{c}{\textbf{EAPI}} &
      \multicolumn{1}{c}{\textbf{Supports \t{IUSE} defaults?}} \\
      \midrule
      0                 & No  \\
      1, 2, 3, 4, 5, 6  & Yes \\
      \bottomrule
    \end{tabular}
\end{centertable}

\ChangeWhenAddingAnEAPI{6}
\begin{centertable}{EAPIs supporting various ebuild-defined variables}
    \label{tab:optional-vars-table}
    \begin{tabular}{lll}
      \toprule
      \multicolumn{1}{c}{\textbf{EAPI}} &
      \multicolumn{1}{c}{\textbf{Supports \t{PROPERTIES}?}} &
      \multicolumn{1}{c}{\textbf{Supports \t{REQUIRED\_USE}?}} \\
      \midrule
      0, 1, 2, 3        & Optionally & No  \\
      4, 5, 6           & Yes        & Yes \\
      \bottomrule
    \end{tabular}
\end{centertable}

\subsection{EAPI}

An empty or unset \t{EAPI} value is equivalent to \t{0}. Ebuilds must not assume that they will get
a particular one of these two values if they are expecting one of these two values.

The package manager must either pre-set the \t{EAPI} variable to \t{0} or ensure that it is unset
before sourcing the ebuild for metadata generation. When using the ebuild for other purposes, the
package manager must either pre-set \t{EAPI} to the value specified by the ebuild's metadata or
ensure that it is unset.

If any of these variables are set to invalid values, the package manager's behaviour is undefined;
ideally, an error in one ebuild should not prevent operations upon other ebuilds or packages.

If the EAPI is to be specified in an ebuild, the \t{EAPI} variable must be assigned to precisely
once. The assignment must not be preceded by any lines other than blank lines or those that start
with optional whitespace (spaces or tabs) followed by a \t{\#} character, and the line containing
the assignment statement must match the following regular expression:
\begin{verbatim}
^[ \t]*EAPI=(['"]?)([A-Za-z0-9+_.-]*)\1[ \t]*([ \t]#.*)?$
\end{verbatim}

The package manager must determine the EAPI of an ebuild by parsing its first non-blank and
non-comment line, using the above regular expression. If it matches, the EAPI is the substring
matched by the capturing parentheses (\t{0} if empty), otherwise it is \t{0}. For a recognised
EAPI, the package manager must make sure that the \t{EAPI} value obtained by sourcing the ebuild
with bash is identical to the EAPI obtained by parsing. The ebuild must be treated as invalid if
these values are different.

\subsection{Keywords}
\label{sec:keywords}

Keywords are used to indicate levels of stability of a package on a respective architecture
\t{arch}. The following conventions are used:
\begin{compactitem}
\item \t{arch}: Both the package version and the ebuild are widely tested, known to work and not
    have any serious issues on the indicated platform. This is referred to as a \i{stable keyword}.
\item \t{\textasciitilde arch}: The package version and the ebuild are believed to work and do
    not have any known serious bugs, but more testing is required before the package version is
    considered suitable for obtaining a stable keyword. This is referred to as an \i{unstable
    keyword} or a \i{testing keyword}.
\item No keyword: It is not known whether the package will work, or insufficient testing has
    occurred.
\item \t{-arch}: The package version will not work on the architecture.
\end{compactitem}
The \t{-*} keyword is used to indicate package versions which are not worth trying to test on
unlisted architectures.

An empty \t{KEYWORDS} variable indicates uncertain functionality on any architecture.

\subsection{\t{RDEPEND} value}
\label{sec:rdepend-depend}

\featurelabel{rdepend-depend} In EAPIs listed in table~\ref{tab:rdepend-depend-table} as having
\t{RDEPEND=DEPEND}, if \t{RDEPEND} is unset (but not if it is set to an empty string) in an ebuild,
when generating metadata the package manager must treat its value as being equal to the value of
\t{DEPEND}.

When dealing with eclasses, only values set in the ebuild itself are considered for this behaviour;
any \t{DEPEND} or \t{RDEPEND} set in an eclass does not change the implicit \t{RDEPEND=DEPEND} for
the ebuild portion, and any \t{DEPEND} value set in an eclass does not get treated as being part of
\t{RDEPEND}.

\ChangeWhenAddingAnEAPI{6}
\begin{centertable}{EAPIs with \t{RDEPEND=DEPEND} default}
    \label{tab:rdepend-depend-table}
    \begin{tabular}{ll}
      \toprule
      \multicolumn{1}{c}{\textbf{EAPI}} &
      \multicolumn{1}{c}{\textbf{\t{RDEPEND=DEPEND}?}} \\
      \midrule
      0, 1, 2, 3        & Yes \\
      4, 5, 6           & No  \\
      \bottomrule
    \end{tabular}
\end{centertable}

\section{Magic Ebuild-defined Variables}

The following variables must be defined by \t{inherit} (see section~\ref{sec:inherit}), and may be
considered to be part of the ebuild's metadata:

\begin{description}
\item[ECLASS] The current eclass, or unset if there is no current eclass. This is handled magically
    by \t{inherit} and must not be modified manually.
\item[INHERITED] List of inherited eclass names. Again, this is handled magically by \t{inherit}.
\end{description}

\note Thus, by extension of section~\ref{sec:metadata-invariance}, \t{inherit} may not be used
    conditionally, except upon constant conditions.

The following are special variables defined by the package manager for internal use and may or may
not be exported to the ebuild environment:

\begin{description}
\item[DEFINED\_PHASES] \featurelabel{defined-phases} A space separated arbitrarily ordered list of
phase names (e.\,g.\ \t{configure setup unpack}) whose phase functions are defined by the ebuild or
an eclass inherited by the ebuild. If no phase functions are defined, a single hyphen is used
instead of an empty string. For EAPIs listed in table~\ref{tab:defined-phases-table} as having
optional \t{DEFINED\_PHASES} support, package managers may not rely upon the metadata cache having
this variable defined, and must treat an empty string as ``this information is not available''.
\end{description}

\note Thus, by extension of section~\ref{sec:metadata-invariance}, phase functions must not be defined
based upon any variant condition.

\ChangeWhenAddingAnEAPI{6}
\begin{centertable}{EAPIs supporting \t{DEFINED\_PHASES}}
    \label{tab:defined-phases-table}
    \begin{tabular}{ll}
      \toprule
      \multicolumn{1}{c}{\textbf{EAPI}} &
      \multicolumn{1}{c}{\textbf{Supports \t{DEFINED\_PHASES}?}} \\
      \midrule
      0, 1, 2, 3        & Optionally \\
      4, 5, 6           & Yes        \\
      \bottomrule
    \end{tabular}
\end{centertable}

% vim: set filetype=tex fileencoding=utf8 et tw=100 spell spelllang=en :

%%% Local Variables:
%%% mode: latex
%%% TeX-master: "pms"
%%% LaTeX-indent-level: 4
%%% LaTeX-item-indent: 0
%%% TeX-brace-indent-level: 4
%%% fill-column: 100
%%% End:
